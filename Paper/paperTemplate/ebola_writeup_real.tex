%%%%%%%%%%%%%%%%%%%%%%%%%%%%%%%%%%%%%%%%%
% Journal Article
% LaTeX Template
% Version 1.1 (25/11/12)
%
% This template has been downloaded from:
% http://www.LaTeXTemplates.com
%
% Original author:
% Frits Wenneker (http://www.howtotex.com)
%
% License:
% CC BY-NC-SA 3.0 (http://creativecommons.org/licenses/by-nc-sa/3.0/)
%
%%%%%%%%%%%%%%%%%%%%%%%%%%%%%%%%%%%%%%%%%

%----------------------------------------------------------------------------------------
%	PACKAGES AND OTHER DOCUMENT CONFIGURATIONS
%----------------------------------------------------------------------------------------

\documentclass[twoside]{article}

\usepackage{amsmath}
\usepackage{graphicx}
\usepackage{lipsum} % Package to generate dummy text throughout this template

\usepackage[sc]{mathpazo} % Use the Palatino font
\usepackage[T1]{fontenc} % Use 8-bit encoding that has 256 glyphs
\linespread{1.05} % Line spacing - Palatino needs more space between lines
\usepackage{microtype} % Slightly tweak font spacing for aesthetics

\usepackage[hmarginratio=1:1,top=32mm,columnsep=20pt]{geometry} % Document margins
\usepackage{multicol} % Used for the two-column layout of the document
\usepackage{hyperref} % For hyperlinks in the PDF

\usepackage[hang, small,labelfont=bf,up,textfont=it,up]{caption} % Custom captions under/above floats in tables or figures
\usepackage{booktabs} % Horizontal rules in tables
\usepackage{float} % Required for tables and figures in the multi-column environment - they need to be placed in specific locations with the [H] (e.g. \begin{table}[H])

\usepackage{lettrine} % The lettrine is the first enlarged letter at the beginning of the text
\usepackage{paralist} % Used for the compactitem environment which makes bullet points with less space between them

\usepackage{abstract} % Allows abstract customization
\renewcommand{\abstractnamefont}{\normalfont\bfseries} % Set the "Abstract" text to bold
\renewcommand{\abstracttextfont}{\normalfont\small\itshape} % Set the abstract itself to small italic text

\usepackage{titlesec} % Allows customization of titles
\renewcommand\thesection{\Roman{section}}
\titleformat{\section}[block]{\large\scshape\centering}{\thesection.}{1em}{} % Change the look of the section titles
%----------------------------------------------------------------------------------------
%	TITLE SECTION
%----------------------------------------------------------------------------------------

\title{\vspace{-15mm}\fontsize{14pt}{12pt}\selectfont\textbf{Modeling the 1995 Ebola Outbreak, \\Democratic Republic of the Congo}} % Article title

\author{
\large
\textsc{Robert Micatka}\\ % Your name
\normalsize University of Michigan \\ % Your institution
}


%----------------------------------------------------------------------------------------

\begin{document}


\maketitle 

\date{}

%----------------------------------------------------------------------------------------
%	ABSTRACT
%----------------------------------------------------------------------------------------

\begin{abstract}

	Using the compartmental epidemiological SIR model as well as an enhanced version to simulate the 1995 Ebola outbreak in
	 the Bandundu region, Democratic Republic of the Congo. Two models are tested, the first is a basic SIR model and the
	 second uses two different infectious classes, a high-risk and low-risk populations to attempt to elucidate any effect, if any,
	that close contact with infected patients, such as that experienced by health-care workers, had on the 
	outbreak size overall as well as within those classes.
 
\end{abstract}

%----------------------------------------------------------------------------------------
%	ARTICLE CONTENTS
%----------------------------------------------------------------------------------------

\begin{multicols}{2} % Two-column layout throughout the main article text

\section{Introduction}

\lettrine[nindent=0em,lines=3]{T} he Ebola virus is an exceedingly deadly (the 1975 outbreak
	had a ~90\% fatality rate \cite{World Health Organization}) viral hemorrhagic fever and is named after the Ebola River 
	in the Democratic Republic of the Congo where it was discovered during an outbreak in 1975.
	The namesake symptom of this disease is the coagulopathy the host shows, which can lead to
	hemorrhaging from the mucous membranes as well as hematemesis (vomiting blood) and
	hemoptysis (coughing blood). This diesase has had several outbreaks in the years since it
	was discovered, mostly in the Democratic Republic of the Congo although other nearby regions have also experienced
	outbreaks. While other species besides humans can be infected by Ebola this is thought to be incidental and not as a 
	source of the virus. There are no confirmed natural reservoirs of the virus, although some research has suggested fruit 
	bats as a probable source \cite{World Health Organization}.\\

	There is no known vaccine for Ebola with the only prevention is to avoid contact with an
	infected individual's bodily fluids. However, this prevention method requires modern medical
	equiptment and procedures which are scarce in the region hit most often by this disease. \\

	This disease has struck in only small numbers and although the fatality rate is extraordinarly 
	high the incidence of the disease is not. This means that the model created had a limited data-set
	and thus inaccuracies are expected. However, the model should be able to shed some light on
	a potential future outbreak and what could be expected from such an outbreak.

%------------------------------------------------

\section{Methods}

	The Susceptible-Infected-Recovered (SIR) Model is a type of compartmental epidemiological model
	that compartmentalizes a population into three distinct groups. This model was chosen over other
	compartmental models because no individual during an epidemic has recovered and become reinfected.
	Population demography was not used due to the short time-span of the outbreaks, the population
	change during the outbreak time will not affect the outcome noticeably. The coupled differential equations
	used for the SIR model are as follows:\\
		\begin{align*}
			& {dS \over dt} = -\beta S I	\\
			& {dI \over dt }= \beta S I - \gamma I	\\
			& {dR \over dt} = \gamma I\\
			& N = S(t) + I(t) +  R(t)
		\end{align*}
	A two-class SIR	model was explored in due to the data consisting of two seperate group of individuals,
	health care workers and non-health care workers. The health care workers would come into contact with
	infected individuals more often than the non-health care workers and so are at a higher risk of
	infection. However, unlike other high-risk groups of other diseases, health care workers do not
	have a higher contact rate with uninfected non-health care workers and so do not neccessarily
	increase the infection rates of the lower-risk group. The coupled differential equations for the
	two-class SIR model are as follows:\\
		\begin{align*}
			& {dS_H \over dt} = -(\beta_{HH} S_H I_H + \beta_{HL} S_H I_L)	\\
			& {dS_L \over dt} = - \beta_{LL} S_L I_L \\
			& {dI_H \over dt} = \beta_{HH} S_H I_H + \beta_{HL} S_H I_L - \gamma I_H	\\
			& {dI_L \over dt} = \beta_{LL} S_L I_L - \gamma I_L \\
			& {dR_H \over dt} = \gamma I_H	\\
			& {dR_L \over dt} = \gamma I_L	\\
			& S(t) =  S_H(t) + S_L(t) \\
			& I(t) =  I_H(t) + I_L(t) \\
			&R(t) =  R_H(t) + R_L(t) \\
			&  N = S(t) + I(t) +  R(t)
		\end{align*}

	The data used in this model is from surveillance case-reports as analyzed by Khan, et al[3].
	This incidence data consisted of the number of new cases during the course of the outreak
	broken down into health care workers and non-health care workers. \\ 

	The data was analyzed by calculating the outbreak over size (using a recovery rate calculated by
	 Astacio et. al. from a previous outbreak [Astacio]) and then applying a linear-regression model using 
	MATLAB software in order to create starting points for the estimation of the appropriate parameters for
	the SIR and the two-class SIR models used.\\

	 The linear-regression model consisted of logarithmic-normalizing the data and finding the slope of the
	 early data points. After the starting parameters were found they were optimized using utilizing
	 MATLAB's minimum-search function. The coupled differential equations were solved in order
	to get a continuous function for the range of the data. This was performed using MATLAB's built in
	ordinary differential equations solver, ode15s.\\

	After the model was completed the mean-square error was computed between the calculated infected function and
	the data for the total population for both the SIR and the SIR with two classes as well as for each of 
	the infectious classes in the SIR with two classes model. This allows for a quantitative assement of a 
	model's accuracy.

%------------------------------------------------



\section{Results}

	The models were run using parameters from the data itself and estimated parameters as described in
	above in the methods section. The graph of the infected individuals from the data is overlayed 
	with both models in Figure 1.\\

	Both models behave similarily, however examining the mean-squared error (MSE) the
	simple SIR model outperforms the two-class SIR model, with calculated MSEs of 60.4386 and 67.4716 respectively.\\
	The infected individual data for the non-health care workers overlayed with the two-class SIR model 
	is shown in Figure 2.\\
	
	The non-health workers are modelled more accurately than the overall population
	in the previous graph. The calculated MSE was 42.9596, much better than the MSEs calculated above.\\
	The infected individual data for health care workers overlayed with the two-class SIR model is
	shown in Figure 3.\\

	The health workers are modelled more accurately than the overall population, similar to the
	improvement seen in the previous graph. The calculated MSE was 42.1551, the graph does not look
	as "good" but the fringes are accurately modelled leading to a low MSE.


%------------------------------------------------

\section{Discussion}

	Overall the SIR and two-class SIR models accurately simulated the data although errors did occur.
	Suprisingly overall the two-class SIR model performed worse than the SIR model on the overall data
	when comparing the MSEs of each. However, the individual classes performed better (ie had a lower MSE) 
	than the overall population model.\\

	The results suggest that the health care workers during this outbreak were at a greater risk than
	the general public because when the contact rates were adjusted results correlated better. This means
	that in the future, stricter protective measures for health care workers should be a priority. While this
	would not a have a large impact on the outbreak size, as workers have low contact rates with uninfected
	individuals, the health care workers themselves would be better protected.\\

	While the models performed well a better fit could have been achieved had more data been available.
	In the future a different model could be tested such as an SEIR model which takes into account a
	latent class in order to account for carriers that are not yet symptomatic. Other possible improvements
	could be better estimation of parameters using more sophisticated methods.


\end{multicols}

\newpage

\section{Figures}




\newpage

%----------------------------------------------------------------------------------------
%	REFERENCE LIST
%----------------------------------------------------------------------------------------

\begin{thebibliography}{99} % Bibliography - this is intentionally simple in this template

\bibitem{World Health Organization} "Ebola haemorrhagic fever." {\em World Health Organization.}
		Web. 23 Apr 2013. <\url{http://www.who.int/mediacentre/factsheets/fs103/en/}>.
	
\bibitem{Khan} Khan, et al "The Reemergence of Ebola Hemorrhagic Fever, Democratic Republic of the Congo, 1995."
	{\em Journal of Infectious Diseases.} (1999): 76-86. Web. 23 Apr. 2013.

\bibitem{Astacio} Astacio, Jaime. "Mathematical Models to Study the Oubreaks of Ebola." Web. 23 Apr. 2013.
		<\url{http://mtbi.asu.edu/files/Mathematical_Models_to_Study_the_Outbreaks_of_Ebola.pdf}>.

\end{thebibliography}

%----------------------------------------------------------------------------------------



\end{document}
